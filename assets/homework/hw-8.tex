\documentclass[11pt]{article}

\usepackage{amsmath}
\usepackage{amssymb}
\usepackage{enumerate,comment}
\usepackage{url}
\usepackage{color}
\usepackage[margin=1.2in]{geometry}
\usepackage{fancyhdr}
\usepackage{xcolor}
\usepackage{hyperref}
\usepackage{cleveref}
\usepackage{mdframed}

\pagestyle{fancy}
\setlength{\headheight}{20pt}
\setlength{\headsep}{20pt}
\fancyhead[L]{\small{CS 276, Fall 2024}}
\fancyhead[R]{\small{Prof. Sanjam Garg}}

\newcommand\custombox[2]{%%
    \fbox{\rule{#1}{0pt}\rule[-0.5ex]{0pt}{#2}}}

\newcommand\answerbox{%%
    \fbox{\rule{1in}{0pt}\rule[-0.5ex]{0pt}{4ex}}}

\newtheorem{theorem}{Theorem}[section]
\newtheorem{definition}[theorem]{Definition}
\newtheorem{corollary}[theorem]{Corollary}
\newtheorem{lemma}[theorem]{Lemma}
\newtheorem{claim}[theorem]{Claim}
\newtheorem{fact}[theorem]{Fact}
\newtheorem{conjecture}[theorem]{Conjecture}
\newtheorem{assumption}[theorem]{Assumption}
\newenvironment{solution}{\color{blue}\noindent{\bf Solution}\hspace*{1em}}{\qed\medskip}
\newcommand{\proof}{\noindent{\bf Proof. }} %% To begin a proof write \proof
\newcommand{\qed}{\mbox{}\hspace*{\fill}\nolinebreak\mbox{$\rule{0.6em}{0.6em}$}} %%to end your proof write $\qed$.

\numberwithin{equation}{section}

% mathbf
\newcommand{\bfA}{\mathbf{A}}
\newcommand{\bfa}{\mathbf{a}}
\newcommand{\bfb}{\mathbf{b}}
\newcommand{\bfc}{\mathbf{c}}
\newcommand{\bfe}{\mathbf{e}}
\newcommand{\bfI}{\mathbf{I}}
\newcommand{\bfM}{\mathbf{M}}
\newcommand{\bfm}{\mathbf{m}}
\newcommand{\bfp}{\mathbf{p}}
\newcommand{\bfr}{\mathbf{r}}
\newcommand{\bfs}{\mathbf{s}}
\newcommand{\bfT}{\mathbf{T}}
\newcommand{\bfu}{\mathbf{u}}
\newcommand{\bfv}{\mathbf{v}}
\newcommand{\bfx}{\mathbf{x}}
\newcommand{\bfy}{\mathbf{y}}

% mathbb
\newcommand{\bbE}{\mathbb{E}}
\newcommand{\bbF}{\mathbb{F}}
\newcommand{\bbG}{\mathbb{G}}
\newcommand{\bbN}{\mathbb{N}}
\newcommand{\bbZ}{\mathbb{Z}}

% mathcal
\newcommand{\cA}{\mathcal{A}}
\newcommand{\cB}{\mathcal{B}}
\newcommand{\cC}{\mathcal{C}}
\newcommand{\cD}{\mathcal{D}}
\newcommand{\cF}{\mathcal{F}}
\newcommand{\cG}{\mathcal{G}}
\newcommand{\cH}{\mathcal{H}}
\newcommand{\cK}{\mathcal{K}}
\newcommand{\cM}{\mathcal{M}}
\newcommand{\cQ}{\mathcal{Q}}
\newcommand{\cR}{\mathcal{R}}
\newcommand{\cS}{\mathcal{S}}
\newcommand{\cX}{\mathcal{X}}
\newcommand{\cY}{\mathcal{Y}}

% Crypto terms
\newcommand{\Setup}{\mathsf{Setup}}
\newcommand{\Gen}{\mathsf{Gen}}
\newcommand{\Enc}{\mathsf{Enc}}
\newcommand{\Dec}{\mathsf{Dec}}
\newcommand{\gen}{\mathsf{Gen}}
\newcommand{\enc}{\mathsf{Enc}}
\newcommand{\dec}{\mathsf{Dec}}
\newcommand{\Sign}{\mathsf{Sign}}
\newcommand{\sign}{\mathsf{Sign}}
\newcommand{\Verify}{\mathsf{Verify}}
\newcommand{\verify}{\mathsf{Verify}}
\newcommand{\MAC}{\mathsf{MAC}}
\newcommand{\mac}{\mathsf{Mac}}
\newcommand{\pk}{\mathsf{pk}}
\newcommand{\sk}{\mathsf{sk}}
\newcommand{\mpk}{\mathsf{mpk}}
\newcommand{\msk}{\mathsf{msk}}
\newcommand{\ct}{\mathsf{ct}}
\newcommand{\secp}{\lambda}

\newcommand{\Rank}{\mathsf{Rank}}
\newcommand{\GF}{\mathsf{GF}}
\newcommand{\msg}{\mathsf{msg}}
\newcommand{\negl}{\mathsf{negl}}
\newcommand{\nonnegl}{\mathsf{nonnegl}}
\newcommand{\st}{\mathsf{st}}
\newcommand{\adv}{\cA}
\newcommand{\hyb}{\mathsf{Hyb}}
\newcommand{\poly}{\mathsf{poly}}

% Math Notation
\newcommand{\getsr}{\stackrel{\$}{\gets}}
\newcommand{\bin}{\{0,1\}}
\newcommand{\bit}{\bin}

\newcommand{\duedate}{Saturday November 16th, 2024 at 8:59pm via Gradescope}

\newif\ifsol
\soltrue

\begin{document}
\section*{CS 276: Homework 8\\ {\small Due Date: \duedate} }

\section{A Proof System for Knowledge of a Preimage}
We will study a general proof system to prove knowledge of a secret preimage $\bfx$ of some public output $\bfy$, for any homomorphic function over a cryptographic group. This protocol generalizes many proof systems, including the Schnorr protocol (that proves knowledge of the discrete log of a group element) and the Chaum-Pedersen protocol (that proves that a given triple of group elements is a DDH triple).

\paragraph{Definitions:} Let $\bbG$ be a cryptographic group of prime order $p$, where $\frac{1}{p} = \negl(\secp)$. Let $d_{in}, d_{out} \in \bbN$ be the dimensions of the input and output spaces, respectively. A function $F$ mapping $\bbZ_p^{d_{in}} \to \bbG^{d_{out}}$ is \textit{homomorphic} if for any $\bfx, \bfx' \in \bbZ_p^{d_{in}}$, $F(\bfx + \bfx') = F(\bfx) \cdot F(\bfx')$. \footnote{Note that the typical group operation for $\bbZ_p$ is addition, and the group operation for $\bbG$ is multiplication, so the homomorphic property simply states that applying the group operation to the inputs before applying $F$ is equivalent to applying the group operation to the outputs after applying $F$.}

The proof system will prove knowledge of a secret preimage $\bfx$ of a public output $\bfy$. An \textit{instance} of the language $L$ is any tuple $(F, \bfy)$ such that $F$ is a homomorphic function mapping $\bbZ_p^{d_{in}} \to \bbG^{d_{out}}$, and $\bfy \in \mathsf{Im}(F)$. The corresponding \textit{witness} is an input $\bfx \in \bbZ_p^{d_{in}}$ such that $F(\bfx) = \bfy$.

For example, if we set $F(\bfx) = g^\bfx$, then we obtain a protocol to prove knowledge of the discrete log $\bfx$ of a given group element $\bfy$. This is essentially the Schnorr protocol. If we set $F(\bfx) = (g^\bfx, h^\bfx)$, then we obtain a protocol to prove that $(g^\bfx, h, h^\bfx)$ are a DDH triple, which is essentially the Chaum-Pedersen protocol.

\paragraph{The Protocol:}
\begin{enumerate}
    \item $P$ samples $\bfa \getsr \bbZ_p^{d_{in}}$, computes $\bfb = F(\bfa)$, and sends $\bfb$ to $V$.
    \item $V$ samples $m \getsr \bbZ_p$ and sends $m$ to $P$.
    \item $P$ computes $\bfc = m \cdot \bfx + \bfa$ and sends $\bfc$ to $V$.
    \item $V$ checks whether
    \[F(\bfc) = \bfy^m \cdot \bfb\]
    If so, $V$ outputs $\mathsf{accept}$. If not, $V$ outputs $\mathsf{reject}$.
\end{enumerate}

\paragraph{Properties of the Protocol:}
Let $P$ and $V$ be the honest prover and honest verifier, who must follow the protocol. Let $P^*$ and $V^*$ be a dishonest prover and verifier, who may deviate from the protocol arbitrarily. Next, the \textit{transcript} of the protocol is $(\bfb, m, \bfc)$, the list of messages sent between the prover and verifier during the protocol.

The protocol should satisfy the following properties:
\begin{itemize}
    \item \textit{Completeness:} If $\bfy = F(\bfx)$, then the protocol between $P(F, \bfy, \bfx)$ and $V(F, \bfy)$ will result in $\mathsf{accept}$ with probability $1$.
    \item \textit{Knowledge Soundness:} There exists an extractor $E$ that runs in expected polynomial time such that for every $F$ and every $\bfy \in \bbG^{d_{out}}$, if $\Pr[\langle P^*, V \rangle(F, \bfy) \to \mathsf{accept}]$ is non-negligible, then $\Pr[F(\bfx') = \bfy : \bfx' \gets E^{P^*}(F, \bfy)]$ is non-negligible as well.
    
    The notation $\langle P^*, V \rangle(F, \bfy) \to \mathsf{accept}$ is the event that the interaction between a dishonest prover $P^*$ and the honest verifier $V$ on inputs $(F, \bfy)$ results in $\mathsf{accept}$. The notation $E^{P^*}$ means that $E$ gets black-box access to $P^*$, which includes the ability to rewind $P^*$.
    \item \textit{Honest-Verifier Zero-Knowledge:} For any valid witness-instance tuple $(\bfx, \bfy, F)$, which satisfies $\bfy = F(\bfx)$, the transcript of the protocol between $P(F, \bfy, \bfx)$ and $V(F, \bfy)$ can be efficiently simulated given only $(F, \bfy)$.
\end{itemize}

\paragraph{Question:} Prove that the protocol given above satisfies completeness, knowledge soundness, and honest-verifier zero-knowledge. Your proof should not require any computational assumptions.

\ifsol
\vspace{5mm}
\begin{solution}
TBD
\end{solution}
\fi

\end{document}