\documentclass[11pt]{article}

\usepackage{amsmath}
\usepackage{amssymb}
\usepackage{enumerate,comment}
\usepackage{url}
\usepackage{color}
\usepackage[margin=1.2in]{geometry}
\usepackage{fancyhdr}
\usepackage{xcolor}
\usepackage{hyperref}
\usepackage{cleveref}
\usepackage{mdframed}

\pagestyle{fancy}
\setlength{\headheight}{20pt}
\setlength{\headsep}{20pt}
\fancyhead[L]{\small{CS 276, Fall 2024}}
\fancyhead[R]{\small{Prof. Sanjam Garg}}

\newcommand\custombox[2]{%%
    \fbox{\rule{#1}{0pt}\rule[-0.5ex]{0pt}{#2}}}

\newcommand\answerbox{%%
    \fbox{\rule{1in}{0pt}\rule[-0.5ex]{0pt}{4ex}}}

\newtheorem{theorem}{Theorem}[section]
\newtheorem{definition}[theorem]{Definition}
\newtheorem{corollary}[theorem]{Corollary}
\newtheorem{lemma}[theorem]{Lemma}
\newtheorem{claim}[theorem]{Claim}
\newtheorem{fact}[theorem]{Fact}
\newtheorem{conjecture}[theorem]{Conjecture}
\newtheorem{assumption}[theorem]{Assumption}
\newenvironment{solution}{\color{blue}\noindent{\bf Solution}\hspace*{1em}}{\qed\medskip}
\newcommand{\proof}{\noindent{\bf Proof. }} %% To begin a proof write \proof
\newcommand{\qed}{\mbox{}\hspace*{\fill}\nolinebreak\mbox{$\rule{0.6em}{0.6em}$}} %%to end your proof write $\qed$.

\numberwithin{equation}{section}

% Crypto terms
\newcommand{\Gen}{\mathsf{Gen}}
\newcommand{\Enc}{\mathsf{Enc}}
\newcommand{\Dec}{\mathsf{Dec}}
\newcommand{\gen}{\mathsf{Gen}}
\newcommand{\enc}{\mathsf{Enc}}
\newcommand{\dec}{\mathsf{Dec}}
\newcommand{\Sign}{\mathsf{Sign}}
\newcommand{\sign}{\mathsf{Sign}}
\newcommand{\Verify}{\mathsf{Verify}}
\newcommand{\verify}{\mathsf{Verify}}
\newcommand{\MAC}{\mathsf{MAC}}
\newcommand{\mac}{\mathsf{Mac}}
\newcommand{\pk}{\mathsf{pk}}
\newcommand{\sk}{\mathsf{sk}}

% mathbb
\newcommand{\bbE}{\mathbb{E}}
\newcommand{\bbF}{\mathbb{F}}
\newcommand{\bbG}{\mathbb{G}}
\newcommand{\bbZ}{\mathbb{Z}}

% mathcal
\newcommand{\cA}{\mathcal{A}}
\newcommand{\cB}{\mathcal{B}}
\newcommand{\cC}{\mathcal{C}}
\newcommand{\cD}{\mathcal{D}}
\newcommand{\cF}{\mathcal{F}}
\newcommand{\cG}{\mathcal{G}}
\newcommand{\cH}{\mathcal{H}}
\newcommand{\cK}{\mathcal{K}}
\newcommand{\cM}{\mathcal{M}}
\newcommand{\cS}{\mathcal{S}}
\newcommand{\cX}{\mathcal{X}}
\newcommand{\cY}{\mathcal{Y}}

% Words
\newcommand{\Rank}{\mathsf{Rank}}
\newcommand{\GF}{\mathsf{GF}}
\newcommand{\msg}{\mathsf{msg}}
\newcommand{\negl}{\mathsf{negl}}
\newcommand{\nonnegl}{\mathsf{nonnegl}}
\newcommand{\st}{\mathsf{st}}
\newcommand{\adv}{\cA}
\newcommand{\hyb}{\mathsf{Hyb}}

% Math Notation
\newcommand{\getsr}{\stackrel{\$}{\gets}}
\newcommand{\bin}{\{0,1\}}
\newcommand{\bit}{\bin}

\newcommand{\duedate}{Friday September 27th, 2024 at 8:59pm via Gradescope}

\newif\ifsol
\soltrue

\begin{document}
\section*{CS 276: Homework 3\\ {\small Due Date: \duedate} }

\section{A Pseudorandom Function Based on Diffie-Hellman}
Let us construct a more efficient variant of the Naor-Reingold PRF. 

\begin{definition}[PRF Construction]\label{def:PRF-construction}
Let $\bbG$ be a cryptographic group of prime order $p$. Let $\ell < p$ be polynomial in $\lambda$. Next, let $s^{*n} = (s_1, \dots, s_n, h)$ be sampled from $\cS^{*n} := \bbZ_p^n \times \bbG$, and let $x^{*n} = (x_1, \dots, x_n)$ be drawn from $\cX^{*n} = [\ell]^n$. Finally, define $F^{*n}:\cS^{*n} \times \cX^{*n} \to \bbG$ as follows:
\begin{align*}
    F^{*n}(s^{*n}, x^{*n}) = 
    \begin{cases}
        1, & \prod_{i \in [n]}(s_i + x_i) = 0\\
        h^{1/\prod_{i \in [n]}(s_i + x_i)}, & \text{else}
    \end{cases}
\end{align*}
\end{definition}
This construction is more efficient than Naor-Reingold's PRF. $F^{*n}$ can handle an input $x^{*n}$ of length $n \cdot \lg(\ell)$ bits, whereas the same seed in the Naor-Reingold PRF would handle inputs of length $n$ bits.

\paragraph{Question:} Prove that the function $F^{*n}$ given in definition \ref{def:PRF-construction} is a secure PRF assuming the $\ell$-DDH assumption (assumption \ref{assumption:DDH-Variant}).

\begin{assumption}[$\ell$-DDH Assumption]\label{assumption:DDH-Variant}
Let $\bbG$ be a cryptographic group of prime order $p$, and let $\ell < p$. Then for any PPT adversary $\cA$, the following two hybrids $\cH_0$ and $\cH_1$ are indistinguishable:
\begin{itemize}
    \item $\cH_0$: The challenger samples $(x, g) \getsr \bbZ_p \times \bbG$ and then gives the adversary $(g, g^x, g^{x^2}, \dots, g^{x^\ell}, g^{1/x})$.
    \item $\cH_0$: The challenger samples $(x, g, h) \getsr \bbZ_p \times \bbG \times \bbG$ and then gives the adversary $(g, g^x, g^{x^2}, \dots, g^{x^\ell}, h)$.
\end{itemize}
Finally, when $x = 0$, then define $g^{1/x} = 1$.
\end{assumption}

\paragraph{Hint:} You may wish to use the following strategy. First, let us define a PRF $f$ over a smaller domain $[\ell]$. Let $f$ take a seed $(s,h) \in \bbZ_p \times \bbG$ and an input $x \in [\ell]$ and output:
\begin{align*}
    f(s,x) = 
    \begin{cases}
        1, & s+x = 0\\
        h^{1/(s+x)}, &\text{else}
    \end{cases}
\end{align*}
First prove that $f$ is a secure PRF when $\ell$ is polynomial in the security parameter $\lambda$.

Second, note that $F^{*n}$ is an $n$-fold composition of $f$, where the output of one invocation of $f$ becomes the $h$-value of the next invocation of $f$.
\[F^{*n}\big((s_1, \dots, s_n, h), (x_1, \dots, x_n)\big) = f((s_n, \dots f((s_2, f((s_1, h), x_1)), x_2) \dots), x_n)\]
Then use a similar proof technique to the one used for Naor-Reingold's PRF to prove that the composition of this small-domain PRF $f$ is also a PRF.

\ifsol
\pagebreak
\begin{solution}
TBD
\end{solution}
\fi

\end{document}