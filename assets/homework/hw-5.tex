\documentclass[11pt]{article}

\usepackage{amsmath}
\usepackage{amssymb}
\usepackage{enumerate,comment}
\usepackage{url}
\usepackage{color}
\usepackage[margin=1.2in]{geometry}
\usepackage{fancyhdr}
\usepackage{xcolor}
\usepackage{hyperref}
\usepackage{cleveref}
\usepackage{mdframed}
% \input{preamble}

\pagestyle{fancy}
\setlength{\headheight}{20pt}
\setlength{\headsep}{20pt}
\fancyhead[L]{\small{CS 276, Fall 2024}}
\fancyhead[R]{\small{Prof. Sanjam Garg}}

\newcommand\custombox[2]{%%
    \fbox{\rule{#1}{0pt}\rule[-0.5ex]{0pt}{#2}}}

\newcommand\answerbox{%%
    \fbox{\rule{1in}{0pt}\rule[-0.5ex]{0pt}{4ex}}}

\newtheorem{theorem}{Theorem}[section]
\newtheorem{definition}[theorem]{Definition}
\newtheorem{corollary}[theorem]{Corollary}
\newtheorem{lemma}[theorem]{Lemma}
\newtheorem{claim}[theorem]{Claim}
\newtheorem{fact}[theorem]{Fact}
\newtheorem{conjecture}[theorem]{Conjecture}
\newtheorem{assumption}[theorem]{Assumption}
\newenvironment{solution}{\color{blue}\noindent{\bf Solution}\hspace*{1em}}{\qed\medskip}
\newcommand{\proof}{\noindent{\bf Proof. }} %% To begin a proof write \proof
\newcommand{\qed}{\mbox{}\hspace*{\fill}\nolinebreak\mbox{$\rule{0.6em}{0.6em}$}} %%to end your proof write $\qed$.

\numberwithin{equation}{section}

% Crypto terms
\newcommand{\Gen}{\mathsf{Gen}}
\newcommand{\Enc}{\mathsf{Enc}}
\newcommand{\Dec}{\mathsf{Dec}}
\newcommand{\gen}{\mathsf{Gen}}
\newcommand{\enc}{\mathsf{Enc}}
\newcommand{\dec}{\mathsf{Dec}}
\newcommand{\Sign}{\mathsf{Sign}}
\newcommand{\sign}{\mathsf{Sign}}
\newcommand{\Verify}{\mathsf{Verify}}
\newcommand{\verify}{\mathsf{Verify}}
\newcommand{\MAC}{\mathsf{MAC}}
\newcommand{\mac}{\mathsf{Mac}}
\newcommand{\pk}{\mathsf{pk}}
\newcommand{\sk}{\mathsf{sk}}
\newcommand{\secp}{\lambda}

% mathbb
\newcommand{\bbE}{\mathbb{E}}
\newcommand{\bbF}{\mathbb{F}}
\newcommand{\bbG}{\mathbb{G}}
\newcommand{\bbN}{\mathbb{N}}
\newcommand{\bbZ}{\mathbb{Z}}

% mathcal
\newcommand{\cA}{\mathcal{A}}
\newcommand{\cB}{\mathcal{B}}
\newcommand{\cC}{\mathcal{C}}
\newcommand{\cD}{\mathcal{D}}
\newcommand{\cF}{\mathcal{F}}
\newcommand{\cG}{\mathcal{G}}
\newcommand{\cH}{\mathcal{H}}
\newcommand{\cK}{\mathcal{K}}
\newcommand{\cM}{\mathcal{M}}
\newcommand{\cQ}{\mathcal{Q}}
\newcommand{\cR}{\mathcal{R}}
\newcommand{\cS}{\mathcal{S}}
\newcommand{\cX}{\mathcal{X}}
\newcommand{\cY}{\mathcal{Y}}

% Words
\newcommand{\Rank}{\mathsf{Rank}}
\newcommand{\GF}{\mathsf{GF}}
\newcommand{\msg}{\mathsf{msg}}
\newcommand{\negl}{\mathsf{negl}}
\newcommand{\nonnegl}{\mathsf{nonnegl}}
\newcommand{\st}{\mathsf{st}}
\newcommand{\adv}{\cA}
\newcommand{\hyb}{\mathsf{Hyb}}
\newcommand{\poly}{\mathsf{poly}}

% Math Notation
\newcommand{\getsr}{\stackrel{\$}{\gets}}
\newcommand{\bin}{\{0,1\}}
\newcommand{\bit}{\bin}

\newcommand{\duedate}{Friday October 18th, 2024 at 8:59pm via Gradescope}

\begin{document}
\section*{CS 276: Homework 5\\ {\small Due Date: \duedate} }


\section{Signature Scheme from CDH}
We will construct a signature scheme that resembles the Schnorr signature scheme and prove it secure given the CDH assumption.

Let $\bbG$ be a cryptographic group of prime order $p$ that is generated by $g$. Also, let $p$ be super-polynomial in the security parameter $\secp$. Let us also define two random oracles $H:\bbG \to \bbG$ and $G:\cM \times \bbG^6 \to \bbZ_p$, where $\cM$ is the message space.
\begin{enumerate}
    \item $\mathsf{\Gen(1^\secp)}$: Sample $x \getsr \bbZ_p$ and compute $y = g^x$. Output $\pk = y$ and $\sk = x$.
    \item $\Sign(\sk, m)$: To sign a message $m \in \cM$, sample $k \getsr \bbZ_p$ and compute the following:
    \begin{align*}
        u &= g^k\\
        h &= H(u)\\
        z &= h^\sk\\
        v &= h^k\\
        c &= G(m, g, h, \pk, z, u, v)\\
        s &= k + c \cdot \sk \mod p\\
        \sigma &= (z, s, c)
    \end{align*}
    Output $\sigma$.
    \item $\Verify(\pk, m, \sigma)$: Compute the following:
    \begin{align*}
        u' &= g^s \cdot \pk^{-c}\\
        h' &= H(u')\\
        v' &= h'^s \cdot z^{-c}\\
        c' &= G(m, g, h', \pk, z, u', v')
    \end{align*}
    Output $1$ (accept) if $c = c'$ and $0$ (reject) otherwise.
\end{enumerate}

\begin{definition}[Computational Diffie-Hellman (CDH) Assumption]
The CDH challenger samples $a, b \getsr \bbZ_p$ independently and gives the adversary $(g, g^a, g^b)$. The adversary wins the CDH game if they return $g^{a \cdot b}$. The CDH assumption states that for any PPT adversary, the probability that the adversary wins the CDH game is $\negl(\secp)$.
\end{definition}

\paragraph{Question:} Prove that the signature scheme constructed above is secure in the random oracle model given the CDH assumption.

\vspace{5mm}
\begin{solution}
TBD
\end{solution}


\section{Additively Homomorphic Encryption (AHE)}
Some natural encryption schemes, such as El Gamal encryption, are additively homomorphic\footnote{This is assuming we use the additive notation for operations over the cryptographic group.}, meaning that $\enc(m^{(1)})$ and $\enc(m^{(2)})$ can be combined into a valid encryption of $m^{(1)} + m^{(2)}$ without knowledge of the secret key. It turns out that this property is sufficient to construct public-key encryption. We will show that secret-key additively homomorphic encryption implies public-key encryption.

\begin{definition}[Additively Homomorphic Encryption]\label{def:AHE}
Let $(\Gen, \Enc, \Dec, H_\oplus)$ be four PPT algorithms with message space $\cM = \bit$ and ciphertext space $\cC$. Let $H_\oplus$ map $\cC^{\ell} \to \cC$, for any $\ell = \poly(\secp)$. 

Next, $(\Gen, \Enc, \Dec, H_\oplus)$ is a \textbf{secret-key additively homomorphic encryption (AHE) scheme}\footnote{\textit{Public-key} additively homomorphic encryption is defined similarly, except $(\Gen, \Enc, \Dec)$ are a public-key encryption scheme, $H_\oplus$ takes $\pk$ as input, and $\Enc$ takes $\pk$, instead of $\sk$, as input.} if the following properties are satisfied: 

\begin{itemize}
    \item \textbf{Perfect Correctness:} For any $\ell = \poly(\secp)$ messages $(m^{(1)}, \dots, m^{(\ell)}) \in \bit^\ell$:
    \[\Pr\bigg[\Dec\Big(\sk, H_\oplus\big[\Enc(\sk, m^{(1)}), \dots, \Enc(\sk, m^{(\ell)})\big]\Big) = \sum_{i \in [\ell]} m^{(i)} \mod 2\bigg] = 1\]
    \item \textbf{Compactness:} There exists a polynomial function $m(\cdot)$ such that for any $\ell = \poly(\secp)$ messages $(m^{(1)}, \dots, m^{(\ell)}) \in \bit^\ell$, the length of $H_\oplus\big[\Enc(\sk, m^{(1)}), \dots, \Enc(\sk, m^{(\ell)})\big]$ is upper-bounded by $m(\secp)$.\footnote{Note that $m(\secp)$ is independent of $\ell$.}
    \item \textbf{CPA security:} $(\Gen, \Enc, \Dec)$ constitute a CPA secure encryption scheme.
\end{itemize}
\end{definition}

The following construction builds a public-key encryption scheme $(\Gen', \Enc', \Dec')$ from a secret-key AHE scheme $(\Gen, \Enc, \Dec, H_\oplus)$.
\begin{enumerate}
    \item $\Gen'(1^\secp)$: Compute the following:
    \begin{align*}
        \sk &\leftarrow \Gen(1^\secp)\\
        \ell' &= 4 m(\secp)\\
        r &\getsr \bit^{\ell'} \backslash \{0^{\ell'}\}\\
        X_i &\gets \Enc(\sk, r_i), \quad \forall i \in [{\ell'}]\\
        \pk &= (X_1, \dots, X_{\ell'}, r)
    \end{align*}
    Then output $(\pk, \sk)$.

    \item $\Enc'(\pk, m)$: 
    \begin{enumerate}
        \item Sample $s \in \bit^{\ell'}$ uniformly at random such that $\langle r, s \rangle = m$.\footnote{$\langle r, s \rangle = \sum_{i \in [{\ell'}]} r_i \cdot s_i \mod 2$. We can sample $s$ using rejection sampling: sample $s \getsr \bit^{\ell'}$ and check whether $\langle r, s \rangle = m$. If not, then reject this $s$ and repeat the procedure.}
        \item Let $X_s$ be a tuple of all the $X_i$-values for which $s_i = 1$.
        \item Compute and output $c = H_\oplus(X_s)$.
    \end{enumerate}
    \item $\Dec'(\sk, c)$: Output $\Dec(\sk, c)$.
\end{enumerate}

\paragraph{Question:} Prove that if $(\Gen, \Enc, \Dec, H_\oplus)$ is a secret-key AHE scheme, then $(\Gen', \Enc', \Dec')$ satisfies (1) CPA security and (2) the following notion of perfect correctness:
\[\Pr\big[\Dec'(\sk, \Enc'(\pk, m)) = m\big] = 1, \quad \forall m \in \bit\]

\vspace{5mm}
\begin{solution}
TBD
\end{solution}

\end{document}