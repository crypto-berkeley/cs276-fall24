\documentclass[11pt]{article}

\usepackage{amsmath}
\usepackage{amssymb}
\usepackage{enumerate,comment}
\usepackage{url}
\usepackage{color}
\usepackage[margin=1.2in]{geometry}
\usepackage{fancyhdr}

\pagestyle{fancy}
\setlength{\headheight}{20pt}
\setlength{\headsep}{20pt}
\fancyhead[L]{\small{CS 171, Spring 2024}}
\fancyhead[R]{\small{Prof. Sanjam Garg}}

\newcommand\custombox[2]{%%
    \fbox{\rule{#1}{0pt}\rule[-0.5ex]{0pt}{#2}}}

\newcommand\answerbox{%%
    \fbox{\rule{1in}{0pt}\rule[-0.5ex]{0pt}{4ex}}}

\newtheorem{theorem}{Theorem}[section]
\newtheorem{definition}[theorem]{Definition}
\newtheorem{corollary}[theorem]{Corollary}
\newtheorem{lemma}[theorem]{Lemma}
\newtheorem{claim}[theorem]{Claim}
\newtheorem{fact}[theorem]{Fact}
\newtheorem{conjecture}[theorem]{Conjecture}
\newtheorem{remark}[theorem]{Remark}
\newenvironment{assumption}{\noindent{\bf Assumption}\hspace*{1em}\begin{em}}{\end{em}\medskip}
\numberwithin{equation}{section}

\newcommand{\Gen}{\mathsf{Gen}}
\newcommand{\Enc}{\mathsf{Enc}}
\newcommand{\Dec}{\mathsf{Dec}}
\newcommand{\Z}{\mathbb{Z}}
\newcommand{\cM}{\mathcal{M}}
\newcommand{\cC}{\mathcal{C}}
\newcommand{\cK}{\mathcal{K}}
\newcommand{\Rank}{\mathsf{Rank}}
\newcommand{\F}{\mathbb{F}}
\newcommand{\getsr}{\stackrel{\$}{\gets}}
\newcommand{\A}{\mathcal{A}}
\newcommand{\D}{\mathcal{D}}
\renewcommand{\H}{\mathcal{H}}
\newcommand{\GF}{\mathsf{GF}}
\newcommand{\negl}{\mathsf{negl}}

\newcommand{\adv}{\A}

\newcommand{\hyb}{\mathsf{Hyb}}

\newcommand{\bin}{\{0,1\}}
\newcommand{\bit}{\bin}
\newcommand{\B}{\mathcal{B}}
\newcommand{\duedate}{March 14th, 2024 at 8:59pm via Gradescope}


\newcommand{\MAC}{\mathsf{MAC}}
\newcommand{\gen}{\mathsf{Gen}}
\newcommand{\mac}{\mathsf{Mac}}
\newcommand{\verify}{\mathsf{Verify}}
\newcommand{\C}{\mathcal{C}}

\newenvironment{solution}{\color{blue}\noindent{\bf Solution}\hspace*{1em}}{\qed\medskip}

\begin{document}
\section*{CS 171: Problem Set 6\\ {\small Due Date: \duedate} }

\section{One-Way Functions}
Let $f:\bin^n \rightarrow \bin^n$ be a one-way function, and let
\[g(x) = f(f(x))\] 
Is $g$ necessarily a one-way function? Prove your answer. In your answer, you may use a OWF $h: \bit^{n/2} \to \bit^{n/2}$.\newline

\noindent \emph{Tip:} Your answer should have one of the following forms. Only one of them is possible:
\begin{itemize}
    \item Prove that if $f$ is a OWF, then $g$ is also a OWF.
    \item (1) Construct a function $f$. (2) Prove that $f$ is a one-way function. (3) Then prove that when $g$ is constructed from this choice of $f$, $g$ is not a one-way function.
\end{itemize}
Also, you may cite without proof any theorems proven in discussion or lecture.

\pagebreak

\section{Concatenated Hash Functions}
Let $\mathcal{H}_1 = (\gen_1, H_1)$ and $\mathcal{H}_2 = (\gen_2, H_2)$ be two fixed-length hash functions that take inputs of length ${3n}$ bits and produce outputs of length $n$ bits. Only one of $\mathcal{H}_1$ and $\mathcal{H}_2$ is collision resistant; the other one is not collision-resistant, and you don't know which is which. 

Next, we define two new hash functions $\mathcal{H}_3 = (\Gen_3, H_3)$ and $\mathcal{H}_4 = (\Gen_4, H_4)$ below: \\

\noindent\underline{$\mathcal{H}_3$:}
\begin{enumerate}
    \item $\Gen_3(1^n)$: Sample $s_1 \leftarrow \Gen_1(1^n)$ and $s_2 \leftarrow \Gen_2(1^n)$. Output $s=(s_1, s_2)$.
    \item $H_3^s(x)$: Output $H_1^{s_1}(x) || H_2^{s_2}(x)$.
\end{enumerate}
Note that $H_3^s: \bit^{3n} \to \bit^{2n}$.\\

\noindent\underline{$\mathcal{H}_4$:}
\begin{enumerate}
    \item $\Gen_4(1^n)$: Sample $s_1 \leftarrow \Gen_1(1^n)$ and $s_2 \leftarrow \Gen_2(1^n)$. Output $s=(s_1, s_2)$.
    \item $H_4^s(x)$: Let $x = (x_1, x_2) \in \bit^{3n} \times \bit^{3n}$. Output $H_1^{s_1}(x_1) || H_2^{s_2}(x_2)$.
\end{enumerate}
Note that $H_4^s: \bit^{6n} \to \bit^{2n}$.

\paragraph{Question:} For each of $\mathcal{H}_3$ and $\mathcal{H}_4$, determine whether the hash function is collision-resistant, and prove your answer.

\pagebreak


\section{Hard-Concentrate Predicates}
Let $f:\bit^n \to \bit^n$ be an efficiently computable one-to-one function. Prove that if $f$ has a hard-concentrate predicate\footnote{Hard-concentrate predicates are defined in Katz \& Lindell, 3rd edition, definition 8.4 under the name \emph{hard-core predicate}.}, then $f$ is one-way.

\pagebreak
\end{document}