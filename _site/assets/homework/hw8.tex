\documentclass[11pt]{article}

\usepackage{amsmath}
\usepackage{amssymb}
\usepackage{enumerate,comment}
\usepackage{url}
\usepackage{color}
\usepackage[margin=1.2in]{geometry}
\usepackage{fancyhdr}
\usepackage{xcolor}

\pagestyle{fancy}
\setlength{\headheight}{20pt}
\setlength{\headsep}{20pt}
\fancyhead[L]{\small{CS 171, Spring 2024}}
\fancyhead[R]{\small{Prof. Sanjam Garg}}

\newcommand\custombox[2]{%%
    \fbox{\rule{#1}{0pt}\rule[-0.5ex]{0pt}{#2}}}

\newcommand\answerbox{%%
    \fbox{\rule{1in}{0pt}\rule[-0.5ex]{0pt}{4ex}}}

\newtheorem{theorem}{Theorem}[section]
\newtheorem{definition}[theorem]{Definition}
\newtheorem{corollary}[theorem]{Corollary}
\newtheorem{lemma}[theorem]{Lemma}
\newtheorem{claim}[theorem]{Claim}
\newtheorem{fact}[theorem]{Fact}
\newtheorem{conjecture}[theorem]{Conjecture}
\newtheorem{remark}[theorem]{Remark}
\newenvironment{assumption}{\noindent{\bf Assumption}\hspace*{1em}\begin{em}}{\end{em}\medskip}
\numberwithin{equation}{section}

\newcommand{\Gen}{\mathsf{Gen}}
\newcommand{\Enc}{\mathsf{Enc}}
\newcommand{\Dec}{\mathsf{Dec}}
\newcommand{\gen}{\mathsf{Gen}}
\newcommand{\enc}{\mathsf{Enc}}
\newcommand{\dec}{\mathsf{Dec}}
\newcommand{\Sign}{\mathsf{Sign}}
\newcommand{\sign}{\mathsf{Sign}}
\newcommand{\Verify}{\mathsf{Verify}}
\newcommand{\verify}{\mathsf{Verify}}
\newcommand{\MAC}{\mathsf{MAC}}
\newcommand{\mac}{\mathsf{Mac}}
\newcommand{\pk}{\mathsf{pk}}
\newcommand{\sk}{\mathsf{sk}}

\newcommand{\Z}{\mathbb{Z}}
\newcommand{\cM}{\mathcal{M}}
\newcommand{\cC}{\mathcal{C}}
\newcommand{\cK}{\mathcal{K}}
\newcommand{\Rank}{\mathsf{Rank}}
\newcommand{\F}{\mathbb{F}}
\newcommand{\getsr}{\stackrel{\$}{\gets}}
\newcommand{\A}{\mathcal{A}}
\newcommand{\B}{\mathcal{B}}
\newcommand{\D}{\mathcal{D}}
\newcommand{\G}{\mathcal{G}}
\renewcommand{\H}{\mathcal{H}}
\newcommand{\GF}{\mathsf{GF}}
\newcommand{\negl}{\mathsf{negl}}
\newcommand{\nonnegl}{\mathsf{nonnegl}}
\newcommand{\msg}{\mathsf{msg}}
\newcommand{\st}{\mathsf{st}}
\newcommand{\GG}{\mathbb{G}}
\newcommand{\ZZ}{\mathbb{Z}}
\newcommand{\adv}{\A}

\newcommand{\hyb}{\mathsf{Hyb}}

\newcommand{\bin}{\{0,1\}}
\newcommand{\bit}{\bin}
\newcommand{\duedate}{April 11th, 2024 at 8:59pm via Gradescope}

\newcommand{\C}{\mathcal{C}}

\newenvironment{solution}{\color{blue}\noindent{\bf Solution}\hspace*{1em}}{\qed\medskip}

\begin{document}
\section*{CS 171: Problem Set 8\\ {\small Due Date: \duedate} }

\section{A New Version of CDH (10 Points)}
We will consider a modified version of the CDH (computational Diffie-Hellman) problem in which an adversary is given $g^x$ and asked to compute $g^{x^2}$. We will show that this modified CDH problem is as hard as the regular CDH problem.

\begin{definition}[CDH Game $\mathsf{CDH}(n, \G, \A)$]
$ $
\begin{enumerate}
    \item Inputs: $n$ is the security parameter. $\G$ is an algorithm that generates a group $\GG$ of prime order $q$. $\A$ is a PPT adversary.
    \item The challenger samples $(\GG, q, g) \leftarrow \G(1^n)$ and also samples $x, y \leftarrow \ZZ_q$ independently. Then, the challenger sends to $\A$ the inputs $(\GG, q, g, g^x, g^y)$.
    \item $\A$ outputs $h \in \GG$. If $h = g^{x \cdot y}$, then the output of the game is $1$ (win). Otherwise, the output of the game is $0$ (lose).
\end{enumerate}
\end{definition}

\begin{definition}[Modified CDH Game $\mathsf{mCDH}(n, \G, \B)$]
$ $
\begin{enumerate}
    \item Inputs: $n$ is the security parameter. $\G$ is an algorithm that generates a group $\GG$ of prime order $q$. $\B$ is a PPT adversary.
    \item The challenger samples $(\GG, q, g) \leftarrow \G(1^n)$ and also samples $x \leftarrow \ZZ_q$. Then, the challenger sends to $\B$ the inputs $(\GG, q, g, g^x)$.
    \item $\B$ outputs $h \in \GG$. If $h = g^{\left(x^2\right)}$, then the output of the game is $1$ (win). Otherwise, the output of the game is $0$ (lose).
\end{enumerate}
\end{definition}

\paragraph{Question:} 
\begin{enumerate}
    \item Prove that if there exists a PPT adversary $\A$ for which $\Pr[\mathsf{CDH}(n, \G, \A) \to 1]$ is non-negligible, then there exists a PPT adversary $\B$ for which $\Pr[\mathsf{mCDH}(n, \G, \B) \to 1]$ is non-negligible.
    \item Prove that if there exists a PPT adversary $\B$ for which $\Pr[\mathsf{mCDH}(n, \G, \B) \to 1]$ is non-negligible, then there exists a PPT adversary $\A$ for which $\Pr[\mathsf{CDH}(n, \G, \A) \to 1]$ is non-negligible.
\end{enumerate}
Together, these claims show that the modified CDH problem is hard if and only if the CDH problem is hard.\\

\begin{solution}
TBD
\end{solution}
\pagebreak

\section{Large-Domain CRHFs From Discrete Log (10 Points)}

We saw in lecture\footnote{See lecture 13, slides 19-20.} how to construct a CRHF assuming the discrete log problem is hard. The CRHF maps $\ZZ_q^2 \to \GG$ (where $\GG$ is a cryptographic group of size $q$). In this problem, we will extend the domain to $\ZZ_q^t$ for any $t = \mathsf{poly}(n)$.

\begin{definition}[A Hash Function $\H = (\Gen, H)$]
$ $
\begin{itemize}
    \item $\gen(1^n)$: Run $\G(1^n)$ to obtain $(\GG, q, g)$. Then sample group elements $h_1,\dots, h_{t-1} \gets \GG$ independently and uniformly at random. Then output: 
    \[s : = \big( \GG, q, g, (h_1,\dots, h_{t-1}) \big)\] 
    as the key.
    \item $H^s(x)$ takes input $x = (x_1, \dots, x_t) \in \ZZ_q^{t}$. Then it outputs 
    \[H^s(x_1,\dots, x_t) := g^{x_t} \cdot \prod_{i=1}^{t-1} h_i^{x_i}\]
\end{itemize}
\end{definition}

\paragraph{Question:} Prove that $\mathcal{H}$ is collision-resistant by completing the proof of theorem \ref{thm:dlog-implies-CRHF-security} below.

\begin{theorem}\label{thm:dlog-implies-CRHF-security}
    If the discrete log problem is hard for $\mathcal{G}$, then $\mathcal{H}$ is collision-resistant.
\end{theorem}

\begin{proof}
\begin{enumerate}
    \item \underline{Overview:} Assume for the purpose of contradiction that $\mathcal{H}$ is not collision-resistant. Then there exists a PPT adversary $\A$ that, on a randomly generated $s$, outputs a collision with non-negligible probability. Then we will construct a PPT adversary $\B$ that breaks the discrete log assumption.
    \item $\B$ will embed the discrete log instance into one index $i \in \{1, \dots, t-1\}$ of the CRHF and sample the other indices of the CRHF randomly.\\
    
    \underline{Construction of $\B$:}
    \begin{enumerate}
        \item Receive $(\GG, q, g, h)$ from the challenger.
        \item Sample $i \gets \{1,\dots, t-1\}$, and set $h_i := h$.
        \item For each $j \in \{1,\dots, t-1\} \setminus\{i\}$, randomly choose $a_j \gets \ZZ_q$ and set $h_j := g^{a_j}$.
        \item Run $\A$ on $\big( \GG, q, g, (h_1,\dots, h_{t-1}) \big)$, and receive a collision $(x_1, \dots, x_t)$ and $(x'_1, \dots, x'_t)$.
        \item In this case, $\B$ outputs
        \[y = \custombox{3in}{.5in}\]
        as the discrete log of $h$.
    \end{enumerate}
    \begin{solution}
            
    \end{solution}

    \item \begin{lemma}\label{thm:B-breaks-dlog}
        If $\A$ breaks the collision-resistance of $\mathcal{H}$, then $\B$ solves the discrete log problem with non-negligible probability.
    \end{lemma}
    \begin{proof}\\
    
        \custombox{5in}{.5in}\\
        
        \begin{solution}
        \end{solution}
    \end{proof}\\
    Note: The size of the box above does not indicate the size of the proof. The proof will most likely not fit in the box.
\end{enumerate}
\end{proof}
\pagebreak

\section{Signatures (10 Points)}
Let $\Pi = (\Gen, \Sign, \Verify)$ be a (secure) signature scheme that accepts messages $m \in \bit^n$. We will use $\Pi$ to construct a candidate signature scheme $\Pi'$ that introduces additional randomness into the signing algorithm.\\

\noindent\underline{$\Pi' = (\Gen', \Sign', \Verify')$:}
    \begin{enumerate}
        \item $\Gen'(1^n)$: Same as $\Gen(1^n)$.
        \item $\Sign'(\sk, m)$: 
        \begin{enumerate}
            \item Let $m \in \bit^n$. Then sample $r \leftarrow \bit^n$.
            \item Compute $\sigma_0 = \Sign(\sk, m \oplus r)$ and $\sigma_1 = \Sign(\sk, r)$.
            \item Output $\sigma = (r, \sigma_0, \sigma_1)$.
        \end{enumerate}
        \item $\Verify'(\pk, m, \sigma)$: Output $1$ if $\Verify(\pk, m \oplus r, \sigma_0) = 1$ and $\Verify(\pk, r, \sigma_1) = 1$. Output $0$ otherwise.
    \end{enumerate}

\paragraph{Question:} Indicate whether or not $\Pi'$ is necessarily secure, and prove your answer.\\

\begin{solution}

\end{solution}

\end{document}